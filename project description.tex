%remote edit 2
\documentclass{article}
\usepackage{amsmath}
\usepackage{graphicx}

\begin{document}

\title{On Collaborative Business Process Evolution}

\date{}
\maketitle

\section{Introduction}
Companies merge with each other for many different reasons such as using each other's customers, and capacities in order to get a bigger market share.
The merge of companies can happen in many different forms depending on the lifeline of each of the companies after merge.
For instance, when a large company acquires a start-up, the start-up may be forced to follow the business processes (BPs) of the large company, since the it is not feasible, or even needed, to change the BPs of the large company to adapt to the ones of the start-up.

In the merge, which is the focus of this research, the two companies need to merge their BPs in order to achieve a common set of business process for using in the newly merged company.
In the merge process, each of the BPs of the two companies is either kept intact; discarded completely; or adapted to its corresponding business process in the other company.
The merge process is a collaborative process, since it needs the collaboration between the two companies, and, one the other hand, it is an iterative process not a one-pass process.
That is why we call it \emph{Collaborative Business Process Evolution} (CBPE in short), in this research.

The difference between CBPE and collaborative business process design is (Hodjat to complete**)

A number of challenging questions arise in merging the BPs of two companies.
As an example, consider the following questions:
\begin{itemize}
  \item It is better to design the BPs of the new company from scratch? or is it better to adapt the BPs of the two companies to the BPs of the new company? Which criteria can be used to choose between the two approaches?
  \item Which BPs should be kept, and which ones should be adapted?
  \item What parameters can be used in designing the new BPs?
  \item What roles in the companies are the best candidates for performing the merge process? and what authorities should they have?
  \item What is the best organizational structure for the groups who are responsible for performing the merge task?
  \item How the conflicts in rules, and the processes can be resolved?
  \item If the merge is done in a number of groups, what method should be followed in order to avoid or deal with the intra-group decisions that have side effects on the other groups?
%  \item Is it possible that the newly merged BPs contain ambiguities while the original BPs, from which the merge BP is generated, do not contain ambiguity (for a definition of ambiguity refer to Section \ref{sec:pd}).
\end{itemize}




A problem, which may arise in the newly merged BPs, is the problem of non-executability, which arises in the case that a BP cannot be executed by workflow engines.
The main reason for the arising of this problem is that the merged BP is ambiguous, meaning that it does not have clear execution semantics or its execution semantics suffers from runtime problems such as deadlock, and livelock.
The non-executability problem may even arise in the case that none of the original BPs, from which the new BP is generated, has this problem.
Thus, to guarantee the quality of the merged BPs, we need to deal with the ambiguities that may arise in the merge procedure.


In this research, we aim at studying the CBPE problem.
As our test case, we use the currently ongoing merge of the science faculties of the University of Amsterdam (UvA), and the Vrije Universiteit (VU), which is part of a bigger plan for merging the two universities.



The rest of this report is structured as follows.
In Section \ref{sec:pd}, we define the notation, and standards which we use, precisely define the research questions that we aim to address, and further explain our test case.
In Section \ref{sec:approach}, we explain our approach for answering the research questions, and the metrics that we use to validate our research results.
Section \ref{sec:RW}, presents the related work.
In Section \ref{sec:tp}, we specify the detailed steps for this research together with the time plan.

\section{Problem definition}\label{sec:pd}





\section{Research Approach}\label{sec:approach}

\section{Related work}\label{sec:RW}

\section{Time plan}\label{sec:tp}






























\end{document} 
